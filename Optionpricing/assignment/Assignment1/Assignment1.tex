\documentclass{article}
\usepackage[utf8]{inputenc}
\usepackage{amsmath,amssymb,amsfonts,amsthm,graphicx,enumitem}
\usepackage[parfill]{parskip}
\usepackage{color}
\usepackage{float}
\usepackage[]{algorithm2e}
\DeclareMathOperator*{\argmin}{arg\,min}
\DeclareMathOperator*{\argmax}{arg\,max}
\usepackage[top=1in,bottom=1in,left=1.4in,right=1.4in]{geometry}

\title{ \vspace{-5mm}
        PDE methods for option pricing}         % <-- Edit Homework Number
\author{\Large{Jian Wang}\\      % <-- Edit Name
\\
FSU ID: jw09r}        % <-- Edit Andrew Id

\begin{document}
\maketitle
\section*{Problem }
\large{
\textbf{[Summary]}

\begin{algorithm}[H]
 \KwData{this text}
 \KwResult{how to write algorithm with \LaTeX2e }
 initialization\;
 \While{not at end of this document}{
  read current\;
  \eIf{understand}{
   go to next section\;
   current section becomes this one\;
   }{
   go back to the beginning of current section\;
  }
 }
 \caption{How to write algorithms}
\end{algorithm}

\textbf{[Statement of the problem]}\\
1.Compute the values of the European vanilla put for $E^*=\$10$, $r^*=0.05/yr$, $\sigma^*=0.20/yr$ and with a six month expiry with and without Rannacher smoothing. Report the error as a function of $\Delta S$ and $\Delta \tau$. Compute the greeks, $\Delta$ and $\Gamma$ and their errors. Propose and implement a technique to compute the $v= \partial V/\partial \sigma$ and report on its performance. Test the effects of the outer boundary on the solution in the range $[0,K]$.\\
2. Redo the previous task for the European binary put. In particular, examine the solution for large $\Delta \tau$ and no smoothing\\

\textbf{[Mathematics Tools]}\\
The BSM and boundary condition for European put option is:\\

$\left \{
\begin{tabular}{l}
$V_t+\frac{\sigma^2S^2}{2}V_{SS}+rSV_s - rV=0$\\

$V(0,t)=Ke^{-r(T-t)}$\\

$V(\infty,t)=0$\\

$V(S,T)=max(K-S,0)$
\end{tabular}
\right.$

(1)We use backward Euler for a few n>=2 time steps\\
(2)Use Crank-Nicolson after that: given second order accuracy \\

combine : $\delta_t^+V_j^n +L_h[\alpha V_j^{n+1}+(1-\alpha)V_j^n]=0$\\
          $\alpha =1$ : Back Euler\\
          $\alpha=\frac{1}{2}$ : Crank-Nicolson method

Crank-Nicolson method:\\


\begin{flalign*}
&\frac{f_{ij}-f_{i,j-1}}{\Delta t}+\frac{ri\Delta S}{2}+(\frac{f_{i+1,j-1}-f_{i-1,j-1}}{2\Delta S}+\frac{ri\Delta S}{2}(\frac{f
_{i+1,j}-f_{i-1,j}}{2\Delta S})+\\
&\frac{\sigma^2 i^2 (\Delta S)^2}{4}(\frac{f_{i+1,j-1}-2f_{i,j-1}+f_{i-1,j-1}}{(\Delta S)^2})+\\
&\frac{\sigma^2 i^2 (\Delta S)^2}{4}(\frac{f_{i+1,j}-2f_{i,j}+f_{i-1,j}}{(\Delta S)^2})=\frac{r}{2}f_{i,j-1}+\frac{r}{2}f_{ij}&&
\end{flalign*}

We can  rewrite the above equation as:\\
\begin{flalign*}
-\alpha_i f_{i-1,j-1}+(1-\beta_i)f_{i,j-1}-\gamma_{i}f_{i+1,j-1}=\alpha_i f_{i-1,j}+(1+\beta_i)f_{i,j}+\gamma_{i}f_{i+1,j}
\end{flalign*}
Where: \\
\begin{flalign*}
\quad\quad &\alpha_i=\frac{\Delta t}{4}(\sigma^2i^2-ri)\\
&\beta_i=-\frac{\Delta t}{2}(\sigma^2i^2+r)\\
&\gamma_i=\frac{\Delta t}{4}(\sigma^2i^2+ri)
\end{flalign*}

}
\end{document} 